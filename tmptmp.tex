\documentclass{article}
\usepackage{geometry}
\usepackage{nopageno}
\usepackage{fvextra}
\usepackage{xcolor}
\geometry{
     a4paper,
     total={192mm,280mm},
     left=10mm,
     top=10mm,
}
\begin{document}
% tiny scriptsize footnotesize small normalsize large
\begin{Verbatim}[fontsize=\small,highlightcolor=lightgray,highlightlines={5,38,58,78}]


This document describes the C-ACE code and documentation directory structures as well as other notes.

Code:

    - Console: during development, before hardware, this application was used to test the algorithms
      and other general C-ACE functionality

    - Deprecated: the SHA-1 and DES algorithms were considered for inclusion in the C-ACE before
      InfoGard/NIST indicated those algorithms are no longer supported

    - General: the main, non-hardware files are here. Algorithms, communications protocols, key storage, etc.

    - Library: for virtual instances of the C-ACE, a library with all of its functionality was built here.

    - Math: there are two large number math libraries used - one for the console application, one for the
      hardware. These third party libraries are here.

    - Projects: the uVision projects for the Preboot, Bootloader and Application projects are located here

    - STM: the hardware files are located here. Subdirectories include specific files for the Preboot,
      Bootloader and Application projects

    - STMCube: the STMCube file for the C-ACE is here. This was used during development to verify peripheral
      connection and to determine the clock scalars to achieve the maximum operating frequency.

    - Utilities: two gcc applications are here, a bootload checksum utility and a signing utility. The
      signing utility also contains the functionality of the bootload checksum utility, so thats the
      only one that is likely to be used. The signing utility will create one large Intel hex file for
      JTAG flashing, combining the Preboot, Bootloader and Application hex files as well as the signed
      flash sector which contains the SHA256 value of the application space [this allows the bootloader
      to verify the application and thus jump to the application on power up]. The console printout will
      also show the R and S values for the DSA signature, which allows the application to be updated via
      the host using SPI/I2C commands.


Documentation:

    - API: the communications protocol document

    - Certification: all of the documents InfoGard required during the initial certification process

    - InfoGard: all documents provided by InfoGard to Red Cocoa during the certification process.
      This also includes the algorithm testing vectors.

    - Installation: a document indicating how to load code with the ST Link Utility

    - Post Certification: a start on the documents outlining the code additions since certification

    - Schematics: a HW schematic of the C-ACE development board

    - STMicro: various STMicro data sheets

    - Test: the test plan used during the InfoGard operational testing visit


Build Environment:

    - We used Keil uVision for development. The version information is below.

    - IDE-Version: uVision V5.10.0.2

    - Tool Version Numbers:

         - Toolchain:            MDK-ARM Standard Cortex-M only  Version: 5.10.0.0
         - Toolchain Path:       C:\Keil_v5\ARM\ARMCC\bin\
         - C Compiler:           Armcc.Exe                                V5.04.0.49
         - Assembler:            Armasm.Exe                               V5.04.0.49
         - Linker/Locator:       ArmLink.Exe                              V5.04.0.49
         - Librarian:            ArmAr.Exe                                V5.04.0.49
         - Hex Converter:        FromElf.Exe                              V5.04.0.49
         - CPU DLL:              SARMCM3.DLL                              V5.10.0.0
         - Dialog DLL:           DCM.DLL                                  V1.10.0.0
         - Target DLL:           STLink\ST-LINKIII-KEIL_SWO.dll
         - Dialog DLL:           TCM.DLL                                  V1.14.1.0

Other notes:

    - The flash starts at address 0x08000000 and is organized as 4 16-KB sectors, 1 64-KB sector,
      and 9 128-KB sector. In order to use the smaller sectors for parameters and keys storage
      and thus lower the flash erase time, a preboot project exists and resides in the first 16-KB
      sector. All this does is execute a jump to the first 128-KB sector which contains the bootloader
      image.

    - The flash is organized as such:

         - 16KB  Block  1:   Preboot application
         - 16KB  Block  2:   Stay in bootloader indication flag [for firmware updating]
         - 16KB  Block  3:   Application signature and length
         - 16KB  Block  4:   Non-key parameters (RSIs, PINs, etc)
         - 64KB  Block:      Keys (traffic, authentication, keysets)
         - 128KB Block  1:   Bootloader
         - 128KB Block  2:   Empty (potential bootloader expansion)
         - 128KB Block  3:   Application block 1
         - 128KB Block  4:   Application block 2
         - 128KB Blocks 5-9: Empty


\end{Verbatim}
\end{document}
